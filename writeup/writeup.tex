%%%% ijcai18.tex

\typeout{IJCAI-18 Instructions for Authors}

% These are the instructions for authors for IJCAI-18.
% They are the same as the ones for IJCAI-11 with superficical wording
%   changes only.

\documentclass{article}
\pdfpagewidth=8.5in
\pdfpageheight=11in
% The file ijcai18.sty is the style file for IJCAI-18 (same as ijcai08.sty).
\usepackage{ijcai18}

% Use the postscript times font!
\usepackage{times}
\usepackage{xcolor}
\usepackage{soul}
\usepackage[utf8]{inputenc}
\usepackage[small]{caption}

% the following package is optional:
%\usepackage{latexsym} 

% Following comment is from ijcai97-submit.tex:
% The preparation of these files was supported by Schlumberger Palo Alto
% Research, AT\&T Bell Laboratories, and Morgan Kaufmann Publishers.
% Shirley Jowell, of Morgan Kaufmann Publishers, and Peter F.
% Patel-Schneider, of AT\&T Bell Laboratories collaborated on their
% preparation.


\title{Poker Project - Team 47: Modelling Poker Players}

\author{
AUYOK Sean, CAI Jiaxiu, CHIK Cheng Yao, Joyce YEO Shuhui, ZHUANG Yihui
\\ 
National University of Singapore\\
%
sean.auyok,e0201975,chikchengyao,joyceyeo,yihui@u.nus.edu
}
% If your authors do not fit in the default space, you can increase it 
% by uncommenting the following (adjust the "2.5in" size to make it fit
% properly)
% \setlength\titlebox{2.5in}

\begin{document}

\maketitle

\section{Introduction}

In a stochastic game like Limit Poker, how far can we exploit imperfect information to maximise our gains? 

Win margin is important, not just about winning.

Our poker agent is also placed under a time constraint for every action, so 

limitations to making decisions

\begin{enumerate}
	\item Functionality, the agent should be rational and be at least able to recognise statistically optimal hole cards which are worth investing in. However, chance alone does not make our agent a winner. In Poker, a core strategy is the ability to model and subsequently predict how an opponent is behaving. This enhancement allows our agent to gain an advantage over randomness, and rake the opponent's money.
	
	\item
	\item Why did you choose this implementation (i.e. why should the reader be interested?)
	\item What is the result that you achieved? (i.e. why should the reader believe you?)
\end{enumerate}

\section{Modelling Game State}

\subsection{Abstracting Card Values}

card isomorphism magic
monte carlo
bucketing strategies and how we generated the bucketing
percentile

generate the win rate table by forcing both players to reveal their cards
OR use a mathematical approach

cfr
spne of counter-factual regret
regret = heuristic

\section{Modelling Opponent Behaviour}

The current research on poker strategies suggests that there is no 1-king in poker tournaments. No one strategy is able to exploit all other strategies in the strategy space. The strategy graph can be described as being non-transistive, where $s_1$ dominating $s_2$ and $s_2$ dominating $s_3$ could still suggest that $s_3$ dominates $s_1$.

In that case then, an optimal agent would have to adapt to the opponent play style and find an element in the set of strategies that would dominate the opponent strategy.  There are two tasks here then, firstly to abstract the strategy states for the agent, and then to approximate the opponent's strategy state based on the observed play.

\subsection{Characterising Play Styles}
A simple way to characterise player strategies is to consider betting behaviour when the player has bad hands and good hands. 

When a player receives a bad hand, they could continue playing or choose to fold. This behaviour can be defined on a tightness-looseness scale. A tight player only plays a small percentage of their hands, and folds otherwise. On the other hand, a loose player would choose to take risks and make bets based on the potential of the hand. Loose play is called bluffing, deceives the opponent into over-estimating the agent's hand strength. The opponents may fold as a result of this observation.

Theoretically, a tight play aims to reduce losses in the case of bad hands. However, a very tight play would also mean that the chances to observe the opponent's behaviour is diminished.

When a player receives a good hand, they could call/ check (keep the pot size stable) or raise their bets. This behaviour can be defined on a passiveness-aggressiveness scale. A passive player keeps their bets low and stable. On the other hand, an aggressive player will actively make raises to increase the game stakes. Passive play is another form of deceit, which leads opponents to under-estimate the hand strength, thereby continuing to place bets and raise the pot amount.

An aggressive play style hopes to maximise the winnings when the hands are strong. However, an over-aggressive play will also encourage opponents to fold, which again discards opportunities to observe opponent behaviour.

An opponent's strategy at one point in time can then be represented as a 2-tuple of (tightness, aggressiveness). Note that this tuple merely represents an instantaneous strategy, and the opponent could change strategies over time because

\begin{enumerate}
	\item The opponent adapts their play to our agent's play style
	\item The opponent employs a team-based strategy where a coach sends out different players to the table depending on the play style our agent uses.
\end{enumerate}

\subsubsection{Heuristic for Tightness and Aggressiveness}

The opponent tightness can be approximated using the folding rate over multiple games. 

The drawback to using the folding-rate heuristic is that a large number of games is needed to make a prediction. By the time these observations are made, the opponent may have already adopted a different play style.

The opponent aggressiveness can be approximated using the raising rate. We make the assumption that when the opponent has a weak hand, they will only call or fold, and raising is only the result of relatively strong hands.

Our agent incorporates these heuristics into its modelling of the opponent by saving the opponent's action history each round.

\subsubsection{Approximating Towards Opponent Strategy}

Given an opponent strategy, we hypothesise that approximating our strategy to be similar to the opponent would minimise our losses. We consider extreme cases to illustrate this point. 

First, consider an opponent that is very tight in play, always folding until they receive a good hand. Suppose on the other hand, that we adopt a loose play and choose to make bets even when our hands are not as ideal. Suppose then, that the opponent obtains a very good hand $h_1$ and we obtain a very bad hand $h_2$. Our looseness will lead us to contribute an amount $x_1$ to the pot before eventually folding or losing during the showdown. By pure chance, there is an equally probability that the two hands are reversed, so the opponents has the very bad hand $h_2$ and we hold on the the very good hand $h_1$. Since the opponent has a tighter play, they would have a lower probability of putting money into the pot, and a higher probability of folding. We let the amount contributed by the opponent to the pot be $x_2$. Clearly, $x_2$, the expected loss by the tight opponent < $x_1$, the expected loss by our loose agent. In the case of tightness, it then makes sense to approximate our tightness to be close to that of the opponent.

Alternatively, consider an opponent that is very aggressive in play, always raising when they receive a good hand. Suppose on the other hand that we adopt a passive play and choose to call or check even when our hands are very strong. (expound on this)

\subsubsection{Randomising Strategy}

The exploitability of opponent behaviour should be symmetric, so our agent has to acknowledge the opponent can similar extract patterns from our play style. It would then be beneficial to either mask our play style or change our play style during the game.

Our agent model generates a 3-tuple for the probability of playing each action (fold, call, raise). For example, one of the tuples generated could be (0.2, 0.3, 0.5). We note here that the probability of raising is the most significant, so we can deduce that our player hand is strong.

Some researchers employ purification techniques to overcome abstraction coarseness. These poker agents prefer the higher-probability actions and ignore actions that are unlikely to be played. In particular, Ganzfried and Sandholm found full purification to be the most effective. The full purification technique let the agent play the most-probable action with probability 1.

However, such a strategy ... makes it predictable?? In particular, the deceit techniques of bluffing and trapping would be less effective. 

\subsection{Predicting Play Styles}

What game states can we use to approximate play styles?
Tightness - Folding Rate
Aggressiveness - Raising Rate

\section{Training}

dqn


\section*{Acknowledgments}

\appendix

\section{\LaTeX{} and Word Style Files}\label{stylefiles}

%% The file named.bst is a bibliography style file for BibTeX 0.99c
\bibliographystyle{named}
\bibliography{ijcai18}

\end{document}

