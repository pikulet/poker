%%%% ijcai18.tex

\typeout{IJCAI-18 Instructions for Authors}

\documentclass{article}
\pdfpagewidth=8.5in
\pdfpageheight=11in
% The file ijcai18.sty is the style file for IJCAI-18 (same as ijcai08.sty).
\usepackage{ijcai18}
\usepackage{times}
\usepackage{xcolor}
\usepackage{soul}
\usepackage[utf8]{inputenc}
\usepackage[small]{caption}

% the following package is optional:
%\usepackage{latexsym} 

\title{Poker Project - Team 47: Modelling Poker Players}

\author{
AUYOK Sean, CAI Jiaxiu, CHIK Cheng Yao, Joyce YEO Shuhui, ZHUANG Yihui
\\ 
National University of Singapore\\
sean.auyok,e0201975,chikchengyao,joyceyeo,yihui@u.nus.edu
}

\begin{document}

\maketitle

\section{Introduction}

In the quest to understand the human mind, scientists have delved into creating Artificial Intelligence (AI) agents in more structured environments like games. AI agents have long established their dominance in games, perhaps most famously in chess with Deep Blue's win over Garry Kasparov in 1996. The capabilities of AI extend into games with imperfect information, and AI agents have beaten the top human experts. 

An interesting example would be Texas Hold'em Poker, a complex game including elements of bluff from both players. More interestingly, it is not enough to merely win the opponent, but also to win by as large a margin as possible. This paper focuses on our team's development of a poker AI for Heads Up Limit Texas Hold'em played under strict time controls. 

Given the limited time for our poker agent to decide on the next action, we employed various optimisation strategies such as using abstraction to reduce the game complexity. We also discuss how we can exploit imperfect information to maximise our gains in a stochastic game.

When designing our agent, the prime functionality was for the agent to be rational and able to recognise statistically optimal hands to invest in. Furthermore, our agent should be able to efficiently model the game state. In particular, we focus on how our agent abstracts card values, which is critical in processing all the different types of cards and combinations that could occur. A good abstraction allows for less calculation by our agent, which is important for the agent to play within the time controls.

Upon recognising strong and weak hands, our agent has to be able to make the right decision (whether to fold, call or bet). In order to give our agent the ability to make such decisions without being predictable, we programmed our agent with Counterfactual Regret Minimisation (CFR). CFR allows our agent to quickly find Nash Equilibria and make decisions based on a probability distribution.

However, calculation alone does not make our agent a winner. We trained our agent to have the ability to model and subsequently predict how an opponent is behaving. This enhancement allows our agent to gain an advantage over randomness, and maximise our gains from our opponent's money.

\begin{enumerate}
	\item Why did you choose this implementation (i.e. why should the reader be interested?)
	\item What is the result that you achieved? (i.e. why should the reader believe you?)
\end{enumerate}

\section{Modelling Game State}
Poker has a plethora of game states an agent can tap on, such as the hand strength, pot size, money left to name a few. Given the complexity of poker then, it would be infeasible to use all the game states in training our agent.

\subsection{Hand Strength}
The primary heuristic in a rational poker agent would be the hand strength, because that is how poker's end state is eventually evaluated. The player with the higher hand strength clears the pot. There is extensive research on evaluators hand strength, such as CactusKev, TwoPlusTwo and 7-card.

These evaluators aim to classify cards into groups of similar play styles, which are called buckets. Since hand strengths (HS) change across streets as community cards are revealed, the classification key is typically a function of the current hand strength, and the potential hand strength.

$f(n) = HS_{curr}(n) + \Sigma(P(HS_{new}(n)) * HS_{new}(n))$

\subsubsection{Card Isomorphism}
To effective evaluate hand strengths, it is imperative to simplify the hands using abstraction techniques like card isomorphism. Card isomorphism relies on the fact that the card's suit has no inherent value. It is only useful to know whether the two hole cards have the same suit. For example, A$\clubsuit$ A$\clubsuit$ has the same winning chance as A$\spadesuit$ A$\spadesuit$. The strategy for these two sets of hole cards is the same.

Card isomorphism can reduce the number of game states in the pre-flop stage from ${52 \choose 2} = 1326$ to $13 \times 13 = 169$ states. Each state is identified using the value of the cards, and a boolean indicator if they are the same suit.

\subsubsection{Percentile Bucketing}
However, card isomorphism does not simplify our the game complexity enough. We employ a second layer of abstraction, card bucketing, to group cards of similar strengths together. Two sets of hole cards like K$\heartsuit$ Q$\diamondsuit$ and Q$\diamondsuit$ J$\heartsuit$ can be handled with the same strategy.

In bucketing, hands can move into different buckets depending on the community cards that have been shown. Johansen suggests using the bucketing sequence in designing the agent, and considers the bucketing history of a hand over the course of a game. However, we chose to consider each street of the game as being independent and did not incorporate this into our agent.

Ganzfried and Sandholm found that using too many buckets meant that the training algorithm could not converge sufficiently fast, eventually setting on a 8-12-4-4 bucket size. We simplified their model and chose to use 5 buckets in all streets for our agent.

Our heuristic for bucketing is the expected win rate of a hand. We used a Monte-Carlo simulation to generate these win rates. In each street (pre-flop, flop, turn, river), we ran a Monte-Carlo Simulation. 1000 different hands were generated each time, with the hands consisting of two pairs of hole cards and 0, 3, 4 or 5 community cards depending on the street. For each of these 1000 hands, 500 Monte-Carlo simulations were run to determine the win rate of the hand.

We used a percentile-win-rate clustering method, so the first bucket comprises the 20\% of cards with the lowest win rates. Effectively, each bucket is identified by a minimum and maximum expected win rate.

can probably put a table here

\section{Modelling Opponent Behaviour}

The current research on poker strategies suggests that there is no 1-king in poker tournaments. No one strategy is able to exploit all other strategies in the strategy space. The strategy graph can be described as being non-transistive, where $s_1$ dominating $s_2$ and $s_2$ dominating $s_3$ could still suggest that $s_3$ dominates $s_1$.

In that case then, an optimal agent would have to adapt to the opponent play style and find an element in the set of strategies that would dominate the opponent strategy.  There are three tasks here then, firstly to abstract the strategy states for the agent, secondly to approximate the opponent's strategy state based on the observed play and lastly to apply a dominating strategy over the opponent's play.

\subsection{Characterising Play Styles}
A simple way to characterise player strategies is to consider betting behaviour when the player has weak hands and strong hands. 

When a player receives a weak hand, they could continue playing or choose to fold. This behaviour can be defined on a tightness-looseness scale. A tight player only plays a small percentage of their hands, and folds otherwise. On the other hand, a loose player would choose to take risks and make bets based on the potential of the hand. Loose play is called bluffing, and deceives the opponent into over-estimating the agent's hand strength. The opponents may fold as a result of this observation.

Theoretically, a tight play aims to reduce losses in the case of weak hands. However, a very tight play would also mean that the chances to observe the opponent's behaviour is diminished.

When a player receives a strong hand, they could call/ check (keep the pot size stable) or raise their bets. This behaviour can be defined on a passiveness-aggressiveness scale. A passive player keeps their bets low and stable. On the other hand, an aggressive player will actively make raises to increase the game stakes. Passive play, also known as slow-playing or trapping, is another form of deceit which leads opponents to under-estimate the hand strength, thereby continuing to place bets and raise the pot amount.

An aggressive play style hopes to maximise the winnings when the hands are strong. However, an over-aggressive play will also encourage opponents to fold, which again discards opportunities to observe opponent behaviour.

An opponent's strategy at one point in time can then be represented as a 2-tuple of (tightness, aggressiveness). Note that this tuple merely represents an instantaneous strategy, and the opponent could change strategies over time because

\begin{enumerate}
	\item The opponent adapts their play to our agent's play style
	\item The opponent employs a team-based strategy where a coach sends out different players to the table depending on the play style our agent uses.
\end{enumerate}

\subsection{Predicting Play Styles}
Given that we decide to model player behaviour using aggressiveness and looseness, we need to extract specify features of the game state to feed into our neural network. In the given engine, the interaction with the opponent comes from the betting actions and the showdown. However, not every game ends in a showdown so using card information from the showdown may not be as effective. Hence, our heuristics for player tightness and aggressiveness is derived from the betting actions of folding and raising.

The opponent tightness can be approximated using the folding rate over multiple games. The drawback to using the folding-rate heuristic is that a large number of games is needed to make a prediction. By the time these observations are made, the opponent may have already adopted a different play style.

The opponent aggressiveness can be approximated using the raising rate. We make the assumption that when the opponent has a weak hand, they will only call or fold, and raising is only the result of relatively strong hands.

Our agent incorporates these heuristics into its modelling of the opponent by saving the opponent's action history each round.

\subsection{Deriving a Dominating Strategy}

\subsubsection{Approximating Strategy}
Given an opponent strategy, we hypothesise that approximating our strategy to be similar to the opponent would minimise our losses. We consider extreme cases to illustrate this point. Let $h_1$ denote a very strong hand and $h_2$ denote a very weak hand, $h_1$ > $h_2$.

First, consider our loose agent playing against a tight opponent. Let the opponent have $h_1$ and let our player have $h_2$. Our looseness means we will contribute an expected amount $x_1$ to the pot before folding or losing during the showdown. By pure chance, there is an equal probability that the two hands are reversed, the opponents has $h_2$ and we hold on to $h_1$. Since the opponent has a tighter play, they would have a lower probability of contributing money into the pot, and a higher probability of folding. Let $x_2$ be the opponent's expected loss. Clearly, $x_2$ < $x_1$. In the case of tightness, approximating our tightness to be close to that of the opponent will reduce our losses on weak hands.

Alternatively, consider our passive agent against an aggressive opponent.  Let the opponent have $h_1$ and let our player have $h_2$. The aggressive opponent will raise the pot amount, and eventually wins an amount $y_1$ during the showdown or when we have folded. Similarly, we have an equal chance to have $h_1$ while our opponent has $h_2$. However, our more passive play will instead opt to keep the pot size stable by calling or checking. In the end, we will win a pot amount $y_2$. Since there was less money put into the pot, then $y_2$ < $y_1$. Matching the opponent's aggressiveness, then, allows us to maximise our agent winnings.

The above analysis disregards the effects of bluffing and trapping, and the risk-taking behaviours which usually manifest with moderate-strength hands.

\subsubsection{Randomising Strategy}

The exploitability of opponent behaviour is symmetric, so our agent has to acknowledge that the opponent can similarly extract patterns from our play style. It would then be beneficial to either mask our play style or change our play style during the game.

Our agent model generates a 3-tuple for the probability of playing each action (fold, call, raise). For example, one of the tuples generated could be (0.2, 0.3, 0.5). We note here that the probability of raising is the most significant, so we can deduce that our player hand is strong. The actual magnitude of the probabilities is determined by our player aggressiveness and tightness, and the evaluated hand strength.

Some researchers employ purification techniques to overcome abstraction coarseness. These poker agents prefer the higher-probability actions and ignore actions that are unlikely to be played. In particular, Ganzfried and Sandholm found full purification to be the most effective. The full purification technique let the agent play the most-probable action with probability 1.

However, we argue against purification because noise in our player behaviour can disrupt the opponent's attempt in identifying patterns in our play style. Yakovenko et al's poker agent player against human players using a fixed play style, and after 100/ 500 hands, the human player was able to recognise mistakes made by the agent and boost their win rate. In particular, the noise supports our deceit techniques of bluffing and trapping.

Besides using noise to generate randomness, we can also change our play style to obfuscate our opponents. Given the same hand as above, our agent may instead generate a 3-tuple like (0, 0.3, 0.7), displaying a significantly higher aggressiveness. We can draw an analogy to local-hill search, where continually exploiting the same strategy may just be leading us to local maxima. The sudden exploration could be less optimal, but may also lead us to an improved local maxima. In coach-based agents, this exploration is analogous to fielding a player the opponent has not met before.

\section{Counter-factual Regret (CFR) Minimisation}

cfr
spne of counter-factual regret
regret = heuristic
shown to approximate equilibrium

\section{Training with a Deep-Q Network}

While we modelled our agent using knowledge-based approaches, we also recognise the importance of using a data-based approach to train our agent. Yakovenko et al posited that a trained model is always more competitive than the model generating the data, since the model was trained to optimise relative to the data. 

Yakovenko's team was developing a generic poker agent trained on a Deep-Q Network to play different forms of poker. We used the same technique of a reinforcement learning algorithm, though our agent is only expected to play Limit Poker.

\subsection{Initial Training}
A basic agent should win against the RandomPlayer and HonestPlayer, which are provided in the engine.


\section*{Acknowledgments}

\appendix

\section{\LaTeX{} and Word Style Files}\label{stylefiles}

%% The file named.bst is a bibliography style file for BibTeX 0.99c
\bibliographystyle{named}
\bibliography{ijcai18}

\end{document}

